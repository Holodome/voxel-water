\chapter{Технологический раздел}

В данном разделе будут рассмотрены особенности реализации программного обеспечения (ПО),
разработанного в конструкторской части работы, и приведены примеры работы программы.

\section{Требования к программному обеспечению}

Программа должна предоставлять следующий функционал:
\begin{itemize}
    \item получение изображения сцены с учетом материалов объектов сцены и освещения;
    \item интерактивное управление камерой;
    \item время отрисовки одного кадра не больше 100мс;
    \item симуляция распространения воксельных частиц воды.
\end{itemize}

Программа должна корректно и оперативно реагировать на все действия пользователя.

\section{Средства реализации}

В качестве графического API для реализации разработанного ПО 
был выбран WebGPU~\cite{WebGPU} -- современный переносимый графический
API.
Он может быть использованным при разработке на любой из платформ, 
поддерживающих классические API (Direct3D~\cite{DirectX12}, 
Metal~\cite{Metal}, Vulkan~\cite{Vulkan}, OpenGLES~\cite{OpenGLES}, WebGL~\cite{WebGL} и др.).

В качестве языка программирования для реализации ПО был выбран Rust -- 
статически типизированный компилируемый язык программирования общего назначения~\cite{Rust}.
Данный выбор обусловлен наличием возможностей 
компиляции на множественное число платформ, в том числе на WASM~\cite{WASM}, наличием
библиотек с поддержкой WebGPU, линейной алгебры.

Для реализации графического интерфейса была выбрана библиотека Dear ImGui~\cite{ImGui}.
Она реализует графический интерфейс в режиме IMGUI (англ. immediate mode GUI). Библиотека написана 
для использования с любым графическим API и представляет способ динамического 
программирования интерфейсов.

ПО реализовано для платформы WASM и нативных MacOS, Linux, Windows. 
Алгоритм отриасовки реализован в вершинном и фрагментном шейдерах на языке WGSL~\cite{WebGPUSL}.

\section{Реализации алгоритмов}

\subsection{Реализация алгогитма генерации случайных чисел}

Для реализации алгоритма трассировки лучей требуется использовать генератор случайных чисел.
В качестве алгоритма генератора случайных чисел был выбран xorshift32~\cite{xorshift}. Это сделано
из-за возможности эффективного выполнения алгоритма на GPU. В листинге~\ref{lst:xorshift.wgsl} представлен
код, выполняющий генерацию случайных чисел.

\includelisting
    {xorshift.wgsl}
    {Реализация генератора случайных чисел в шейдере}

\subsection{Реализация алгоритма быстрого обхода вокселей для трассировки лучей}

В листинге~\ref{lst:voxel.wgsl} приведена реализация алгоритма быстрого 
обхода вокселей для трассировки лучей.

\pagebreak

\includelisting
    {voxel.wgsl}
    {Реализация алгоритма быстрого обхода вокселей для трассировки лучей}

\subsection{Реализация алгоритма трассировки лучей}

В листинге~\ref{lst:trace.wgsl} приведена реализация алгоритма трассировки лучей.
Функция \verb|scatter| выполяет расчет цвета и направления отражения луча.

\includelisting
    {trace.wgsl}
    {Реализация алгоритма трассировки лучей}

\subsection{Реализация алгоритма кеширования обратной репроекции}

В листинге~\ref{lst:cache.wgsl} приведена реализация алгоритма кеширования обратной репроекции.

\includelisting
    {cache.wgsl}
    {Реализация алгоритма кеширования обратной репроекции}

Каждый кадр использует 4 буфера, содержащих следующую информацию о каждом пикселе:
\begin{itemize}
    \item цвет,
    \item вектор нормали,
    \item координаты первого столкновения луча при трассировке,
    \item идентификатор материала.
\end{itemize}

Каждый кадр имеет доступ к буферам прошлого кадра, что позволяет выполнять обратную репроекцию.

\section*{Вывод}

В данном разделе были описаны детали реализации разработанной программы. 
Также был рассмотрен процесс взаимодействия пользователя с программой.
Были приведены примеры использования программы.
