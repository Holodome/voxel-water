\chapter{Технологический раздел}

\section{Требования к программному обеспечению}

Программа должна предоставлять следующий функционал:
\begin{itemize}
    \item загрузка сцены из файла;
    \item вокселизация сцены;
    \item получение изображения сцены с учетом материалов объектов сцены и освещения;
    \item интерактивное управление камерой;
    \item время отрисовки одного кадра не больше 100мс;
    \item симуляция распространения воксельных частиц воды.
\end{itemize}

Программа должна корректно и оперативно реагировать на все действия пользователя.

\section{Средства реализации}

В качестве графического API для реализации разработанного ПО 
был выбран WebGPU~\cite{WebGPU} -- современный переносимый графический
API.
Он может быть использованным при разработке на любой из платформ, 
поддерживающих классические API (Direct3D~\cite{DirectX12}, 
Metal~\cite{Metal}, Vulkan~\cite{Vulkan}, OpenGLES~\cite{OpenGLES}, WebGL~\cite{WebGL} и др.).

В качестве языка программирования для реализации ПО был выбран Rust -- 
статически типизированный компилируемый язык программирования общего назначения~\cite{Rust}.
Данный выбор обусловлен наличием возможностей 
компиляции на множественное число платформ, в том числе на WASM~\cite{WASM}, наличием
библиотек с поддержкой WebGPU, линейной алгебры.

Для реализации графического интерфейса была выбрана библиотека Dear ImGui~\cite{ImGui}.
Она реализует графический интерфейс в режиме IMGUI (immediate mode GUI). Библиотека написана 
для использования с любым графическим API и представляет способ динамического 
программирования интерфейсов любой сложности.

ПО реализовано для платформы WASM и нативных (MacOS, Linux, Windows).

\section{Реализация алгогитма трассировки лучей}

Алгоритм трассировки реализована в 
вершинном и фрагментном шейдерах на языке WGSL~\cite{WebGPUSL}.
В качестве генератора случайных чисел был выбран алгоритм xorshift32~\cite{xorshift}
с временем отрисовываемого кадра в качестве источника энтропии. Это сделано
из-за возможности эффективного выполнения алгоритма на GPU в 
силу параллелазции на уровне инструкций. В листинге~\ref{lst:xorshift.wgsl} представлена 
функции генерации случайных чисел, и инициализация генератора случаных чисел.

\includelisting
    {xorshift.wgsl}
    {Реализация генератора случайных чисел в шейдере}

\section*{Вывод}

В данном разделе были описаны детали реализации разработанной программы. 
Также был рассмотрен процесс взаимодействия пользователя с программой.
Были приведены примеры использования программы.
