\chapter{Исследовательская часть}

\section{Технические характеристики}

Технические характеристики устройства, на котором выполнялось тестирование:
\begin{itemize}
    \item операционная система MacOS Ventura 13.2.1 (22D68);
    \item 16 ГБ оперативной памяти;
    \item процессор Apple M1~\cite{M1};
    \item графический ускоритель Apple M1~\cite{M1}.
\end{itemize}

Графический ускоритель имеет 8 ядер, каждое из которых имеет 16 модулей 
исполнения, каждый из которых содержит 8 АЛУ. Таким образом, 
одновременно могут испольняться 24576 потоков и максимальная вычислительная мощность
равняется 2.4 TFLOPS.

\section{Постановка исследования}

Целью исследования является определение времени отрисовки кадра 
от дальности отрисовки и максимального числа отскоков луча в алгоритме трассировки.

Оба параметра варьируются от 1 до 128.

\section{Средства реализации}

Временем кадара является разность значения процессороного времени на момент начала кадра с 
аналогичным значением прошлого кадра. При замерах сцена остается неизменной, что
минимизирует число накладных расходов. В качестве времени кадра берется
среднее значение за отрисовку 1000 кадров одной сцены.

Замеры проводились на ноутбуке, включенном в сеть 
электропитания. Во время замеров ноутбук был нагружен только 
встроенными приложениями окружения, окружением, а 
также непосредственно тестируемой программой.

\section{Результат исследования}

\section{Вывод}

