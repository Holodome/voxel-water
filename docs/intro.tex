\chapter{Введение}

Термин "Компьютерная графика" обозначает любое использование компьютера для создания 
и манипуляции изображениями. Задачи и области применения компьютерной графики 
сложно охарактеризовать в силу их большого колчичества, однако в общем случае можно выделить 
следующие направления:
\begin{itemize}
    \item \textbf{Моделирование} -- занимается математическим описанием формы и внешнего вида
в формате, пригодным для использования в компьютере;
    \item \textbf{Рендеринг} или \textbf{Отрисовка} -- термин, заимствонный из изобразительного 
искусства, обозначающий создание изображений из компьютерных моделей;
    \item \textbf{Анимация} -- это техника создания иллюзии движения используя последовательность изображений~\cite{FoCG}.
\end{itemize}

Рендеринг -- это основная составляющая компьютерной графики. На самом высоком уровне абстракции рендеринг является процессом преобразования описания трехмерной сцены в изображение. Алгоритмы для создания анимации, геометрического моделирования, нанесения текстур и других областей компьютерной графики должны проходить через некий процесс рендеринга, чтобы их результаты могли быть видны на изображении. Рендеринг стал повсеместным: от кино до игр и далее, он открыл новые горизонты для творческого выражения, развлечения и визуализации.

В первые годы развития этой области исследования в графическом рендеринге сосредоточивались на решении фундаментальных проблем, таких как определение, какие объекты видны из определенной точки обзора. По мере нахождения эффективных решений для этих проблем и доступности более богатых и реалистичных описаний сцен благодаря продолжающемуся прогрессу в других областях графики, современный рендеринг начал включать идеи из широкого спектра дисциплин, включая физику и астрофизику, астрономию, биологию, психологию и изучение восприятия, а также чистую и прикладную математику. Междисциплинарный характер рендеринга является одной из причин, почему это такая увлекательная область исследований.

В последние годы активно стала развиваться фотореалистичная компьютерная графика. Это дисциплина находится на стыке физики и классической компьютерной графики. Высокий реализим изображений требует больших вычислительных мощностей для их получения, что заставляет разработчиков постоянно искать новые, более эффективне способы рендеринга.

Фотореалистичная компьютерная графика теперь повсеместно используется, включая такие области как развлечения, в частности, кино и видеоигры, дизайн продуктов и архитектура. За последнее десятилетие широкое распространение получили физически ориентированные методы визуализации, где точное моделирование физики рассеяния света является основой синтеза изображений. Эти подходы обеспечивают как визуальный реализм, так и предсказуемость~\cite{PBRT3e}.

Цель данной работы -- реализовать программу для построения 
реалистичных изображений трехмерных воксельных сцен в реальном времени,
с возможностями физически-корректной визуализации воды.

Чтобы достигнуть поставленой цели, требуется решить следующие задачи.

\begin{enumerate}
    \item Описать структуру трехмерной сцены, включая объекты, из которых
          состоит сцена и их материалы, и определить способ задания 
          исходных данных;
    \item Выбрать или разработать алгоритмы компьютерной 
          графики, позволяющие визуализировать трехмерную воксельную
          сцену в реальном времени;
    \item Реализовать выбранные и адаптированные алгоритмы построения 
          трехмерной сцены;
    \item Исследовать возможности улучшения производительности программы
          и повышения сложности задаваемых сцен.
\end{enumerate}
