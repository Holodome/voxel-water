\section*{Введение}
\addcontentsline{toc}{section}{Введение}

Компьютерная графика на сегодняшний день вездесуща. В общем случае 
задачу компьютерной графики можно описать как получение изображения 
на экране из описания сцены. 

Под реалистичным ренднерингом имеется в виду построение сцен
с учетом физических свойст света (к примеру, отражения или преломления), 
а также с учетом свойств материалов.

Цель данной работы -- реализовать программу для построения 
реалистичных изображений трехмерных воксельных сцен в реальном времени,
с возможностями физически-корректной визуализации воды.

Чтобы достигнуть поставленой цели, требуется решить следующие задачи:

\begin{enumerate}
    \item Описать структуру трехмерной сцены, включая объекты, из которых
          состоит сцена и их материалы, и определить способ задания 
          исходных данных;
    \item Выбрать или разработать алгоритмы компьютерной 
          графики, позволяющие визуализировать трехмерную воксельную
          сцену в реальном времени;
    \item Реализовать выбранные и адаптированные алгоритмы построения 
          трехмерной сцены;
    \item Исследовать возможности улучшения производительности программы
          и повышения сложности задаваемых сцен.
\end{enumerate}
