\chapter*{Введение}
\addcontentsline{toc}{chapter}{ВВЕДЕНИЕ}

Термин "Компьютерная графика" обозначает любое использование компьютера для создания 
и манипуляции изображениями. Задачи и области применения компьютерной графики 
сложно охарактеризовать в силу их большого колчичества, однако в общем случае можно выделить 
следующие направления:
\begin{itemize}
    \item \textbf{моделирование} -- занимается математическим описанием формы и внешнего вида
в формате, пригодным для использования в компьютере;
    \item \textbf{рендеринг} или \textbf{отрисовка} -- термин, заимствонный из изобразительного 
искусства, обозначающий создание изображений из компьютерных моделей;
    \item \textbf{анимация} -- это техника создания иллюзии движения используя последовательность изображений~\cite{FoCG}.
\end{itemize}

Рендеринг -- это основная составляющая компьютерной графики. 
На самом высоком уровне абстракции рендеринг является процессом 
преобразования описания трехмерной сцены в изображение. Алгоритмы для создания 
анимации, геометрического моделирования, нанесения текстур и других областей 
компьютерной графики должны проходить через некий процесс рендеринга, 
чтобы их результаты могли быть видны на изображении. Рендеринг стал 
повсеместным: от кино до игр и далее, он открыл новые горизонты для 
творческого выражения, развлечения и визуализации~\cite{PBRT3e}.

Цель данной работы -- реализовать программу для построения 
реалистичных изображений трехмерных воксельных сцен в реальном времени,
с возможностью физически-корректной визуализации воды.

Чтобы достигнуть поставленой цели, требуется решить следующие задачи:
\begin{itemize}
    \item описать структуру трехмерной сцены, включая объекты, из которых
          состоит сцена и их материалы;
    \item выбрать алгоритмы компьютерной 
          графики, позволяющие визуализировать трехмерную воксельную
          сцену в реальном времени;
    \item реализовать выбранные алгоритмы построения трехмерной сцены;
    \item исследовать возможности улучшения производительности программы
          и повышения сложности задаваемых сцен.
        % TODO: конкретное исследование
\end{itemize}
