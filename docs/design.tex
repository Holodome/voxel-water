\chapter{Конструкторский раздел}

В данном разделе будут описаны особенности написания программ для графических процессоров, 
будут разработаны алгоритмы, выбранные для решения поставленной задачи.

\section{Особенности написания программ для графических процессоров}

Написание программ для графических процессоров имеет следующие особенности:
\begin{itemize}
    \item получение входных данных в виде последовательности графических примитивов;
    \item разделение программы на несколько этапов;
    \item использование специального языка программирования;
    \item отсутствие прямого доступа к памяти;
    \item высокая вычислительная стоимость условных переходов.
\end{itemize}

Входными данными являются треугольники. При отрисовке сцены используется два треугольника,
определяющих прямоугольник видимости экрана.
Программы, называемые шейдерами (англ. Shader), конвеера (англ. Pipeline) 
графического процессора деляется на несколько этапов. Двумя основными этапами, 
без которых отрисовка не может быть выполнена, это этапы выполнения вершинного (англ. Vertex) и 
фрагментного (англ. Fragment) шейдеров. Вершинный шейдер выполняется для каждой из вершин один
раз, и его результаты линйеного интерполируются. Фрагментный шейдер
выполняется для каждого пикселя экрана и выполняет подсчет цвета этого пикселя.

Алгоритм трассировки лучей реализуется в фрагментном шейдере.

\section{Разработка алгоритмов}

\subsection{Алгоритм отрисовки}

На рисунке~\ref{img:app_algo} приведена схема алгоритма работы программы.
$MAX\_BOUNCE\_COUNT$ -- это целочисленная константа, обозначающая максимальное число
отскоков одного луча. Под цветом понимается
трехмерный вектор в формате RGB в диапозоне $[0, 1)$. $rgb(0,0,0)$ -- функция черного цвета. 

\includeimage
    {app_algo}
    {f}
    {H}
    {0.70\textwidth}
    {Алгоритм отрисовки}

\subsection{Алгоритм кэширования обратной репроекции}

Временная обратная репроекция -- это процесс отображения ранее сгенерированного кадра на 
текущий кадр. Это позволяет повторно использовать информацию или, в случае трассировки 
лучей, накапливать сэмплы (и тем самым сходиться к решению уравнения отображения) даже при движении~\cite{ARTSwRPC}.

На рисунке~\ref{img:rrc} представлена схема алгоритма кэширования обратной
репроекции.

\includeimage
    {rrc}
    {f}
    {H}
    {0.50\textwidth}
    {Алгоритм кеширования обратной репроекции}

\section*{Вывод}

В данном разделе были описаны особенности алгоритмов для выполнения
на графических процессорах, алгоритмы и структуры данных, выбранные 
и разработанные для решения поставленной задачи. 
