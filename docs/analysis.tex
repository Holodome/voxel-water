\chapter{Аналитический раздел}

В данном разделе будут рассмотрены возможножсти отрисовки 
трехмерных изображений в реальном времени, проанализированы особенности трехмерной 
воксельной графики; будет выбран метод решения поставленной задачи.

\section{Выбор оборудования}

Построение трехмерных изображений -- вычислительно сложная, и в то же время, чрезвычайно 
параллельная задача~\cite{DaBPP}. Такие особенности стали причиной появляния специального оборудования 
для выполнения популярных задач компьютерной графики. Такие вычислительные устройства
называются Графическими процессорами (GPU), хотя их применение не ограничивается вычислениями
графических данных. Появление таких устройств вызвано ограничениями в дизайне центральных
процессоров. Они используются для последовательного выполнения кода с большим числом условных 
переходов, на небольшом числе ядер при симметричной многопроцессорности (SMP). И несмотря на 
развитие SIMD парадигмы, позволяющей обрабатывать большее число данных, чем позволяет
последовательное исполнения, ЦПУ не смогли стать универсальными устройствами для рендеринга.
Особенностью графических сопроцессоров ялвяется большое число ядер, способных эффективно 
выполнять тысячи вычислений параллельно~\cite{ACLaG}.

Несмотря на большие возможности параллельного выполнения вычислительных задач,
графические сопроцессоры имеют органиченные возможности условного выполнения программ, в 
частности, рекурсивных, малую скорость памяти, и возможности взаимодействия с ней.

Такие особенности породили целый ряд алгоритмов, оптимизированных для выполняения на графических
сопроцессорах, которые отличаются от аналогичных для процессоров общего назначения. В частности, 
графические сопроцессоры могут иметь специальные аппаратные блоки для ускоренного выполнения алгоритмов трассировки лучей. ~\cite{RTiv}

\section{Выбор алгоритма отрисовки}

123

\subsection{Алгоритм, использующий Z-буффер}\mbox{} \\
123~\cite{Rodzhers}

\subsection{Трассировка лучей}""\\
132~\cite{PBRT3e}

\paragraph{Метод Монте-Карло}""\\
123~\cite{PBRT3e}

\paragraph{Алгоритм кеширования обратной репроекции}""\\
123~\cite{ARTSwRPC}

\section{Выбор алгоритма обхода воксельной сцены}

Воксель (аналогично Пиксель) -- способ представления геомтерии, альтернативный типичному
полигональному. Воксель представляет собой некоторое значение на регулярной решетке в 
трехмерном пространстве. Воксели часто испольуются при анализе медицинских и научных
данных.

В интерактиных графических приложениях, выполняющих отрисовку в реальном времени, воксели 
редко применялись в качестве основного геометрического примитива. Это связано с тем, что
графические сопроцессоры были специально оптимизированы для работы с полигоныльными данными,
и они были более простым способом достичь требуемого качества изображения. 

В последние годы замечается тенденция повышения популярности вокселей в интерактивных приложениях.
Это связано с повышением мощности вычислительной техники, позволяющей выполнять ранее невозможные
вычисления на пользовательских компьютерах. Воксельная графика позволяет достигать большей детализации,
чем возможно при использовании полигонов. 

Главная проблема воксельной графики -- большое число затрачиваемой памяти и медленный доступ к ней.
Поэтому все воксельны алгоритмы фокусируются на ускорении доступа к индивидуальным вокселям, для
использования в алгоритмах отрисовки. Рассмотрим и выберем алгоритм обхода воксельной сцены и 
структуру данных хранения вокселей.

\subsection{Алгоритм быстрого обхода вокселей для трассировки лучей}""\\
123~\cite{AFVTAfRT}

\subsection{KD-дерево}""\\
123~\cite{KDTASfaGR}

\subsection{Разреженное воксельное октодерево}""\\
123~\cite{ESVOAEaI}

\section{Выбор модели освещения}

123

\subsection{Простая закраска}""\\
123~\cite{Rodzhers}

\subsection{Модель освещения Фонга}""\\
123~\cite{IFCGP}

\subsection{Физически-корректная модель}""\\
123~\cite{PBRT3e}

\paragraph{Микрогранные модели}""\\
123~\cite{PBRT3e}~\cite{MMfRrRS}

\section{Вывод}

Было принято решение выполнять разработку для выполнения на графическом сопроцессоре,
поскольку это позволяет выполнять отрисовку значетельно более эффективно, чем на ЦПУ.

В качестве алгоритма трассировки была выбрана обратная трассировка лучей с помощью
метода Монте-Карло, поскольку он имеет
лучшие возможности для получния физически-корректных изображений, чем 
алгоритм, использующий Z-буффер.
Для оптимизации отрисовки был выбран способ кеширования 
обратной репроекции.

В качестве модели освещения была выбрана физически-корректная модель с использованием
микрограней с моделью GGX, поскольку такой вариант предоставляет лучший вариант
получения реалистичных изображений, чем другие модели освещения и модель микрограней Бекерманна~\cite{MMfRtRS}.
